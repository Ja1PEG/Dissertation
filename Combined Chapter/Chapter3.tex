\chapter{Scaling the Game Board: Impact on Game Hardness and Duration}

\section{Introduction: Board Size as a Determinant of Game Hardness}

In the preceding chapter, the influence of snake and ladder \textit{lengths} and the \textit{number} of these entities on the board was systematically examined. It was observed how varying entity lengths, through different distributional approaches, alters the average game time and the overall shape of the game's turn distribution.  Having established the sensitivity of game dynamics to entity lengths, this chapter now turns its attention to another fundamental parameter: the \textit{size} of the game board itself. 

This chapter investigates how scaling the dimensions of the game board, while maintaining a consistent density of snakes and ladders and keeping their individual lengths fixed, impacts key gameplay metrics.  The primary focus will be on understanding how board size influences the \textit{hardness} of the game. In this context, game hardness is operationally defined through two readily quantifiable measures: \textbf{average game time}, representing the typical duration of a play session, and \textbf{probability of winning within a specified number of turns}, reflecting the likelihood of achieving a relatively quick victory.  These metrics provide complementary perspectives on game challenge and player experience, with average game time indicating the overall time investment required and win probability offering insight into the game's pace and potential for swift success.

By systematically varying the board size and analysing the resulting changes in average game time and win probabilities, this chapter aims to elucidate how board dimensions, in conjunction with fixed entity characteristics, shape the mechanical properties and, by extension, the mechanical enjoyability of Snakes and Ladders.  The hypothesis is that increasing board size, even with constant entity density and lengths, will lead to longer average game times, reflecting the greater distance to traverse to reach the goal. However, the precise nature of this scaling relationship, and whether it remains linear or exhibits more complex patterns, remains an open question that this chapter will address through simulation-based analysis.


\section{Methodology: Simulation Setup for Board Size Scaling}

To investigate the impact of board size, a simulation experiment was designed, varying the linear dimension, $n$, of a square Snakes and Ladders board, resulting in board sizes $N = n^2$. The following board sizes were systematically explored: $8 \times 8$ (64 tiles), $10 \times 10$ (100 tiles), $12 \times 12$ (144 tiles), $14 \times 14$ (196 tiles), $16 \times 16$ (256 tiles), $18 \times 18$ (324 tiles), and $20 \times 20$ (400 tiles).  For each board size, a constant density of snakes and ladders was maintained, set at 0.1 entities per tile, meaning the number of snakes and the number of ladders were both calculated as $0.1 \times N$.  Crucially, the individual lengths of all snakes and ladders were also kept \textit{fixed} at 20 tiles, irrespective of board size, to isolate the effect of board dimensions.

For each board size configuration, 10,000 game simulations were conducted using the agent-based simulation program developed in the preceding chapters. In each simulation, the number of turns taken to complete the game, the sequence of tile positions visited by the agent, and the instances where snakes or ladders were triggered were recorded. From these simulation runs, the following key metrics were extracted to quantify game hardness:

\begin{enumerate}
	\item \textbf{Average Game Time (Simulation):}  The mean number of turns taken across 10,000 simulated games for each board size, providing a measure of typical game duration.
	\item \textbf{Probability of Winning within N/2 Turns:} Calculated as the proportion of games completed within $N/2$ turns, where $N$ is the board size, representing the probability of a relatively quick win (within half the number of tiles).
	\item \textbf{Probability of Winning within N/4 Turns:}  Calculated as the proportion of games completed within $N/4$ turns, representing the probability of a very swift win (within a quarter of the number of tiles).
\end{enumerate}

