\chapter*{Abstract}

In contrast to the idea of ``experiential enjoyment" of games, a qualitative attribute that reflects a player's subjective immersion and emotional response, this dissertation investigates ``mechanical enjoyment"—defined as pleasure derived from the game's design and mechanics—through parametric analysis, this study conducts a methodical investigation of interactions between several game parameters in the classic board game of Snakes and Ladders. This study makes use of a mixed-methods approach, combining agent-based simulations and Markov Chain modelling, to examine game dynamics in Snakes and Ladders. The study investigates the effects of entity characteristics such as the numbers of snakes and ladders on the board, or the lengths of snakes and ladders and related metrics, as well as the board size on game duration and win probabilities, going beyond subjective interpretations of game enjoyment. Several kinds of relationships are found, including some surprising results that suggest that larger boards increase the likelihood of faster wins even though they lead to longer game durations on average. For game designers looking to identify how to balance difficulty and enjoyment through the means of setting up the pacing in tabletop games, the Markov Chain model provides to be an effective tool for analysing mechanical enjoyment while minimising computational costs required in agent-based simulations.
\linebreak

\textit{Keywords}: \textit{snakes and ladders, tabletop games, markov chain, agent-based simulations, game design, mechanical enjoyment, game duration, win probability}