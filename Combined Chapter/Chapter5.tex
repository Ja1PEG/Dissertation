\chapter{Scaling the Game Board: Impact on Game Hardness and Duration}

\section{Introduction: Board Size as a Determinant of Game Hardness}

In the preceding chapter, we systematically examined the influence of snake and ladder \textit{lengths} on the dynamics of Snakes and Ladders. We observed how varying entity lengths, through different distributional approaches, alters the average game time and the overall shape of the game's turn distribution.  Having established the sensitivity of game dynamics to entity lengths, we now turn our attention to another fundamental parameter: the \textit{size} of the game board itself. 

This chapter investigates how scaling the dimensions of the game board, while maintaining a consistent density of snakes and ladders and keeping their individual lengths fixed, impacts key gameplay metrics.  Our primary focus will be on understanding how board size influences the \textit{hardness} of the game. In this context, we define game hardness operationally through two readily quantifiable measures: \textbf{average game time}, representing the typical duration of a play session, and \textbf{probability of winning within a specified number of turns}, reflecting the likelihood of achieving a relatively quick victory.  These metrics provide complementary perspectives on game challenge and player experience, with average game time indicating the overall time investment required and win probability offering insight into the game's pace and potential for swift success.

By systematically varying the board size and analyzing the resulting changes in average game time and win probabilities, this chapter aims to elucidate how board dimensions, in conjunction with fixed entity characteristics, shape the mechanical properties and, by extension, the mechanical enjoyability of Snakes and Ladders.  We hypothesize that increasing board size, even with constant entity density and lengths, will lead to longer average game times, reflecting the greater distance to traverse to reach the goal. However, the precise nature of this scaling relationship, and whether it remains linear or exhibits more complex patterns, remains an open question that we will address through simulation-based analysis.


\section{Methodology: Simulation Setup for Board Size Scaling}

To investigate the impact of board size, we designed a simulation experiment varying the linear dimension, $n$, of a square Snakes and Ladders board, resulting in board sizes $N = n^2$. We systematically explored the following board sizes: $6 \times 6$ (36 tiles), $8 \times 8$ (64 tiles), $10 \times 10$ (100 tiles), $12 \times 12$ (144 tiles), $14 \times 14$ (196 tiles), $16 \times 16$ (256 tiles), $18 \times 18$ (324 tiles), and $20 \times 20$ (400 tiles).  For each board size, we maintained a constant density of snakes and ladders, set at 0.1 entities per tile, meaning the number of snakes and the number of ladders were both calculated as $0.1 \times N$.  Crucially, we also kept the individual lengths of all snakes and ladders \textit{fixed} at 20 tiles, irrespective of board size, to isolate the effect of board dimensions.

For each board size configuration, we conducted 10,000 game simulations using the agent-based simulation program developed in the preceding chapters. In each simulation, we recorded the number of turns taken to complete the game, the sequence of tile positions visited by the agent, and the instances where snakes or ladders were triggered. From these simulation runs, we extracted the following key metrics to quantify game hardness:

\begin{enumerate}
	\item \textbf{Average Game Time (Simulation):}  The mean number of turns taken across 10,000 simulated games for each board size, providing a measure of typical game duration.
	\item \textbf{Probability of Winning within N/2 Turns:} Calculated as the proportion of games completed within $N/2$ turns, where $N$ is the board size, representing the probability of a relatively quick win (within half the number of tiles).
	\item \textbf{Probability of Winning within N/4 Turns:}  Calculated as the proportion of games completed within $N/4$ turns, representing the probability of a very swift win (within a quarter of the number of tiles).
\end{enumerate}

In addition to these quantitative metrics, we also generated visualizations to aid in understanding the game dynamics across different board sizes. These visualizations include:

\begin{itemize}
	\item \textbf{Turn Distribution Histograms:} Histograms displaying the probability distribution of game turns for each board size, illustrating the variability and typical range of game durations.
	\item \textbf{Tile Visit Frequency Heatmaps:} Heatmaps visualizing the frequency with which each tile is visited during simulations, highlighting board hotspots and areas of high player traffic.
	\item \textbf{Game Trajectory Plots:} Line plots showing the progression of tile position over turns for a small number of example games, providing a visual representation of game unfolding and typical game paths.
\end{itemize}

The following sections will present and analyse the findings from these simulations, focusing on how average game time and win probabilities scale with increasing board size, and interpreting these trends in the context of game hardness and mechanical enjoyability.