\chapter{Conclusion}
This dissertation systematically explores the game of \textit{Snakes and Ladders}, driven by the ambition to look beyond the subjective interpretations of game enjoyment and establishing a more quantifiable understanding of what we define as ``mechanical enjoyment".  Our understanding of ``mechanical enjoyment" refers to the enjoyment derived by simply the game design and mechanics. By embracing a mixed approach, combining agent-based simulations to build a base and then Markov Chain modelling, we tried to explore the impact of diverse game parameters and design elements on the emergent game dynamics. While the findings mostly agree with our hypotheses, but at times they proved to be counter-intuitive. The interplay between the strategic board design, and inherent probabilistic outcomes was studied, not only within the specific context of \textit{Snakes and Ladders} but also offering broader insights applicable to the wider realm of tabletop games.

Looking at the empirical approach to gaining insight, Chapter 2’s agent-based study investigated the effects of systematically varying the number and lengths of snakes and ladders.  This chapter unveiled a crucial bit of information: while mere jumps in the number of snakes and ladders did not linearly correlate with game duration, extreme disparities in their counts significantly affected a game's duration.  Specifically, configurations with a disproportionately high number of snakes coupled with fewer ladders were demonstrably prone to producing larger outlier game sessions of extended length, indicative of heightened ‘luck dependency’ and potential player frustration. Conversely, increasing $N_l$ was shown to act as a counter-balance, tending to lower game durations and diminish variability, thus promoting a more consistent and predictable player experience.  Furthermore, our investigation into entity lengths revealed that the relationship between $L_s$ and $L_l$, rather than absolute lengths themselves, acted as a bigger modulator of game duration. Wider disparities, particularly scenarios where snakes were designed to be markedly longer than ladders, consistently translated to shorter game durations, shedding light on the intuitive impact of setbacks outweighing the short-cuts on perceived game ``difficulty".

Chapter 3 extended the investigation to examine the impact of scaling the game board, maintaining consistent entity density. Counter-intuitively, and perhaps most strikingly, this section of the research revealed that while larger boards predictably \textit{increase} average game time—a finding consistent with the notion of more tiles to traverse—they simultaneously, and paradoxically, \textit{enhance} the probability of achieving quicker victories within defined turn limits.  This effect was attributed to the proportionally greater number of pathways and expanded tile options inherent in larger board configurations.  Increased board size, in essence, provides players with a more ``diffused" game space, offering a statistically higher likelihood of circumventing clusters of snakes and capitalising on strategically advantageous ladder placements, even within the context of overall prolonged game sessions.  Across both parameter variation and board scaling experiments, the ratio of snakes to ladders consistently emerged as a dominant and readily tunable factor, demonstrably influencing both mean game duration and the probability of achieving a quick win, thereby solidifying its crucial role as a primary modulator of overall game difficulty and player-perceived pacing.

Chapter 4 we moved on from agent-based simulations and focused on constructing a Markov Chain model. This model, constructed to capture the transitions by dice rolls and entity interactions, proved effective in analytically predicting several game metrics.  Upon comparing our insights from the prior chapters, and, juxtaposing model-derived predictions against empirically observed data from agent-based simulations, we saw a compelling similarity across expected game durations, win probabilities within a certain number of turns, and entire game turn distributions.  Beyond just the predictive accuracy, the Markov model distinguished itself through its analytical efficiency, offering a direct and computational method to derive these crucial game metrics. 

The ``ideal" Snakes and Ladders board layout discussed in this study does not mean reducing the overall duration of games, but a carefully balanced set of several design objectives, which may be in conflict with one another. The $\frac{N_s}{N_l}$ ratio stands out for its simplicity as a good design parameter, in that it enables a good balancing of the challenge and the pacing, whereas the board size appears to be a much more complicated parameter: it impacts the overall length of the game, as well as the subtle changes in the statistical probability of the player's victory speed, which is faster but not that often consistent. In the scope of this dissertation, we propose a clear and measurable way of making such critical decisions regarding the design that will lead to further design attempts toward the more appealing, more balanced, and more commercially successful tabletop game iterations.

It is, however, important to acknowledge the limitations both in the methodology and scope of this research. Firstly, the stochasticity that characterises \textit{Snakes and Ladders} and which is suitably captured by probabilistic models also brings along an element of randomness that bounds the predictability. Even though Markov modelling and simulation analyses generate estimates for average game behaviour across many iterations, every single game session will inherently show some kind of variance because that's what the probability of a die roll does.  Secondly, the exploration of board layout design space, while systematic in its parameter variations and board scaling experiments, necessarily remained constrained to a relatively limited subset of all conceivable board configurations.  The research primarily focused on changing the entity counts, lengths, and overall board size, but did not exhaustively explore the potentially significant impact of \textit{specific placements} of snakes and ladders, etc. 

In conclusion, this dissertation has tried to establish that, there is great value in having a systematic and quantifiable approach to understanding the complex dynamics behind simple games like \textit{Snakes and Ladders}. By effectively combining empirical simulation with the analytical power of Markov Chain modelling, the research not only provides practical and actionable insights for game designers seeking to create more engaging and balanced tabletop experiences but also contributes a methodological framework and a more objective, data-driven foundation for future research within the ever-evolving field of game studies, bridging the gap between game mechanics and player enjoyment within tabletop games. Future research should try to study all kinds of interactions that relative features such as $\frac{N_s}{N_l}$ and $L^s_i-L^l_i$ have with other possible metrics which directly have impacts on enjoyment, such as getting stuck between a section of the board due to the nature of the configuration itself. Apart from just looking at the mechanical enjoyment that this dissertation has already looked at, future research should aim to close the distance between qualitative tools to assess experiential enjoyment, and the notion of mechanical enjoyment.