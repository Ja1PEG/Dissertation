\chapter{Introduction}
From the simple delight of childhood board games to the intricate strategies of modern tabletop experiences, games hold a fundamental appeal for humanity. Indeed, play itself, as Peter Gray (2017) argues, is not merely frivolous pastime but a powerful vehicle for learning and development, deeply ingrained in our nature. Games, in their essence, are structured systems that invite players to engage in artificial conflicts constrained by predefined rules, ultimately leading to quantifiable outcomes (Puentedura, 2009). This act of play, this engagement within a rule-bound system, is where the potential for enjoyment resides.  This research proposal aims to explore the factors influencing both game and experiential enjoyability, seeking to develop a comprehensive scoring system that quantifies mechanical enjoyability across various genres and playstyles.  Understanding and quantifying this 'enjoyment', however, becomes a complex undertaking.  As Nicole Lazzaro (2004) emphasises, enjoyment is inherently subjective, influenced by myriad factors ranging from individual preferences and social dynamics to the inherent design and mechanics of the game itself.

This complexity necessitates a multidisciplinary approach, combining insights from ludology, psychology, mathematics, and game design, to comprehensively examine game enjoyability. Tabletop games, encompassing board games, card games, and dice games, offer a particularly rich and tangible space for this exploration. Their tangible nature, the direct manipulation of components, and face-to-face social interaction create a fertile ground for investigating the sources of game enjoyment.  Within this exploration, one must consider the very philosophical underpinnings of what constitutes a game. Bernard Suits, in his seminal work *The Grasshopper: Games, Life and Utopia* (1978), provides one such valuable framework. Suits introduces the concept of the "lusory attitude", the willing acceptance of constitutive rules to engage in activity aimed at achieving a specific state of affairs (the lusory goal), where such rules prohibit the most efficient means of achieving that state. This "lusory attitude" is central to understanding games as distinct from ordinary life, operating within what Johan Huizinga (1938) termed the "magic circle"—a bounded space where different rules and expectations apply.

This dissertation focuses on the specific domain of tabletop games, chosen for their tangible nature and the direct player interaction they allow.  Within this domain, we will specifically examine the game of Snakes and Ladders.

\section{Moving Beyond Subjectivity}
While the allure of games is universally acknowledged, the nature of enjoyment itself remains inherently subjective (Lazzaro, 2004). Existing game review systems, as analysed by Yang and Mei (2010), often grapple with this subjectivity, relying on consumer feedback inherently limited by subjective opinions and the tendency to focus on "search attributes" – features readily apparent before playing – rather than "experiential attributes" – those felt only through gameplay.  Yang and Mei’s (2010) research further reveals that negative reviews disproportionately influence perceptions, and the network effect, where shared experiences amplify enjoyment (or dissatisfaction), further complicates objective assessment.

However, this inherent subjectivity does not negate the need for a more systematic and potentially quantifiable approach to understanding game enjoyment. Indeed, to advance game design and analysis, one must strive to bridge the gap between subjective experience and objective analysis. This research affirms that while experiential enjoyability remains inherently variable, mechanical enjoyability, rooted in the game design's core mechanics (or the rules), can be approached as a more quantifiable construct. By focusing on the design elements and rule systems that structure gameplay, this project aims to develop methods for objectively assessing and potentially predicting the level of enjoyment a game's mechanics might elicit, focusing on what can be termed "mechanical enjoyability" – the enjoyment derived from the inherent design and mechanics of the game system itself.  This "mechanical enjoyability" attributed to the various facets of a game's design refers to a quantitative measure that can potentially be assigned to a game to indicate the level of enjoyment or the utility a player derives from the game.

\section{Snakes and Ladders: A timeless classic}
To ground our exploration of game enjoyability in a tangible example, this dissertation will reference the game of Snakes and Ladders.  As Marcus du Sautoy (2023) illuminates by exploring the mathematical underpinnings of games, Snakes and Ladders is far from being a trivial childhood pastime; it boasts a rich history and a remarkable universality that makes it an ideal case study for understanding fundamental game mechanics and player engagement. Its origins can be traced back to ancient India, where it was known as \textit{Moksha Patam} or \textit{Gyan Chaupar}. The ladders represented virtues like generosity, faith, and humility, while the snakes symbolized vices such as lust, anger, theft, and pride. The ascent and descent on the board mirrored the karmic cycle of life,
illustrating the consequences of good and bad actions in a visually compelling and accessible way.

Over centuries, Moksha Patam travelled beyond India, evolving and adapting as it spread across cultures, a journey that resonates with Sautoy’s (2023) broader narrative of how ideas and concepts traverse geographical and cultural boundaries. By the late 19th century, a Westernised version, ”Snakes and Ladders,” emerged in England and quickly gained popularity worldwide. While the overt moralistic undertones diminished in its global iteration, the core mechanics of chance, progression, setbacks, and the simple pursuit of a defined goal remained intact. This appeal across diverse cultures and time periods, echoing themes of in mathematical and historical contexts, underscores the game’s ability to tap into fundamental aspects of human engagement and enjoyment.

While often perceived as a simple children's game, Snakes and Ladders has also proven to be a rich subject for academic inquiry, particularly within the field of mathematical game analysis.  As early as 1967, Daykin, Jeacocke, and Neal (1967) demonstrated in their work, 'Markov Chains and Snakes and Ladders,' that the game constitutes an "interesting example of a Markov chain," amenable to rigorous mathematical modelling. Researchers have since employed Markov Chains to analyse various aspects of Snakes and Ladders, including the probability distributions of game length (Zin Mar Tun, 2021) and the calculation of expected playing time (Althoen et al., 1993; Daykin et al, 1967).  Althoen et al. (1993) estimated the average game length to be approximately 39 moves through both analytical calculations and computer simulations, providing a benchmark figure for understanding the typical duration of a game of Snakes and Ladders.

Snakes and Ladders, in its simplicity, offers a microcosm of the broader challenges in quantifying game enjoyability. Its mechanics are easily grasped - the roll of a die dictates movement, and predetermined snakes and ladders introduce elements of both fortune and misfortune. Yet, even within this seemingly straightforward system, players experience a range of emotions: anticipation with each dice roll, frustration upon encountering a snake, elation when climbing a ladder, and the ultimate satisfaction of reaching the final square. The game's accessibility and widespread familiarity make it an excellent lens through which to examine how even basic game mechanics, governed by chance and simple rules, can generate engaging and emotionally resonant player experiences. By analysing Snakes and Ladders through the framework of mechanical and experiential enjoyability, one can begin to isolate and understand the core design elements that contribute to the enduring appeal of tabletop games, and potentially, games more broadly.  Therefore, this dissertation will use Snakes and Ladders as a central case study to explore and quantify mechanical enjoyability.

\section{Game Typologies}
Games can be categorised in various ways, reflecting different modes of play and player motivations. The modes of playing a game can range anywhere from competitive, cooperative, solo, casual, exploratory, and role-playing, each incentivising a distinct playstyle.  Caillois and Barash (2001) propose four main groups into which different game styles fall:
\begin{itemize}
	\item \textbf{Agon:} Games embodying competition, where players compete directly against one another, exemplified by games like Chess. These games often elicit “Hard Fun” (Lazzaro, 2004), centred around mastery and challenge.
	\item \textbf{Alea:} Games of chance and luck, introducing uncertainty and randomness that influence outcomes. Snakes and Ladders itself falls into this category, driven by die rolls. These games may elicit “Easy Fun” (Lazzaro, 2004), focused on simple enjoyment and relaxation.
	\item \textbf{Mimicry:} Games encouraging make-believe and role-playing, where participants take on new personas and immerse themselves in fictional universes, such as Dungeons and Dragons. These games often tap into “Altered State” enjoyment (Lazzaro, 2004), offering escapism and transformative experiences.
	\item \textbf{Ilinx} Games that induce altered states of consciousness and provide experiences beyond ordinary perception.
\end{itemize}

\subsection{Deterministic vs. Stochastic Games}
Mathematically speaking, games can be broadly classified based on the predictability of their outcomes and the role of chance: Deterministic and Stochastic games. Deterministic games have entirely predictable outcomes determined by player actions and game rules. In games like Chess, assuming perfect play, the optimal move can be determined from any board configuration, making the game deterministic in its outcome.  Stochastic games, conversely, incorporate elements of uncertainty or randomness, leading to less predictable outcomes. Snakes and Ladders is a prime example of a stochastic game.  Gameplay is infused with randomness through die rolls, making the game's progression probabilistic rather than predetermined. This research focuses on analysing stochastic games, using Snakes and Ladders as a primary case study due to its inherent randomness and suitability for probabilistic modelling.  

\section{Game Studies and Analytical Approaches}


The academic study of games is a diverse and interdisciplinary field, spanning user experience design, social context analysis, mathematical game theory, and more (Vlachopoulos \& Makri, 2017). Key perspectives within game studies include narratology, focusing on games as narrative experiences, and ludology, emphasising game rules and structures (McManus \& Feinstein, 2006). While narratology examines the story and narrative elements within games, ludology prioritises the systematic analysis of game mechanics and player interaction with these systems.  Nicole Lazzaro’s (2004) ``Four Keys to More Emotion'' framework, derived from user experience research, further categorises game enjoyment into “Hard Fun”, “Easy Fun”, “Altered State”, and the “People Factor”, highlighting the diverse sources of player engagement and providing a user-centric perspective.

Mathematical analysis provides another crucial lens for examining games. Game theory, a branch of applied mathematics, offers tools to study strategic decision-making in competitive situations (von Neumann, Morgenstern, \& Rubinstein, 1944).  Furthermore, methods like combinatorial analysis and, significantly for this research, Markov Chains, have been applied to analyse game mechanics and dynamics.  For instance, Nilsson (2020) used Markov Chains and simulations to explore strategies in Monopoly, while Raposo and Lamont (2023) employed mathematical analysis to investigate the Royal Game of Ur. These diverse analytical approaches, ranging from theoretical frameworks to mathematical modelling, provide a rich toolkit for objectively investigating game mechanics and their impact on player experience.

\section{Dissertation Structure}
To guide the reader through this exploration, the dissertation follows a structured trajectory, progressing from foundational concepts to empirical analysis and analytical validation.  Beginning with the establishment of essential frameworks and a review of existing approaches to game analysis and enjoyability, the research moves towards an empirical investigation of game dynamics. This empirical phase employs agent-based simulations to quantify the influence of specific game parameters on player experience within a chosen game.  Building upon these empirical findings, the investigation then advances to explore the impact of a key design element – the game board's scale – through further simulation-based analysis.  Subsequently, the research shifts to a more theoretical plane, developing and validating a mathematical model to analytically derive key game metrics and provide a comparative validation against the earlier empirical results.  Finally, the dissertation concludes by aggregating the key insights gained throughout this exploration, discussing their implications for game design and the broader field of game studies, and suggesting potential avenues for future research and inquiry. 
