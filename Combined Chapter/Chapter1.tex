\chapter{Introduction}
From the simple delight of childhood board games to the intricate strategies of modern tabletop experiences, games hold a fundamental appeal for humanity. Play itself, as Peter \textcite{grayWhatExactlyPlay2017} argues, is not merely a frivolous pastime but a powerful vehicle for learning and development, deeply ingrained in our nature. Games, in their essence, are structured systems that invite players to engage in artificial conflicts constrained by predefined rules, ultimately leading to quantifiable outcomes \autocite{puenteduraGameLearnIntroduction}. This act of play, this engagement within a rule-bound system, is where the potential for enjoyment resides. However, understanding and quantifying this `enjoyment' becomes a complex undertaking.  As Nicole \textcite{lazzaroWhyWePlay2004a} emphasises, enjoyment is inherently subjective, influenced by myriad factors ranging from individual preferences and social dynamics to the inherent design and mechanics of the game itself.

Given this subjectivity, a challenge arises - how could we systematically analyse game enjoyment? Such a question necessitates a multidisciplinary approach, combining insights from ludology, psychology, mathematics, and game design to examine game enjoyment comprehensively. Tabletop games, encompassing board games, card games, and dice games, offer a vibrant and tangible space for this exploration. The direct manipulation of components and face-to-face social interaction creates a fertile ground for investigating the sources of game enjoyment. Within this exploration, one must consider the philosophical underpinnings of what constitutes a game. Bernard Suits, in his seminal work ``The Grasshopper: Games, Life and Utopia" \textcite{suitsGrasshopperGamesLife1978}, provides one such valuable framework that positions enjoyment within the structure of gameplay itself. Suits introduces the concept of the ``lusory attitude", the willing acceptance of constitutive rules to engage in activity aimed at achieving a specific state of affairs (the lusory goal), where such rules prohibit the most efficient means of achieving that state. This ``lusory attitude" is central to understanding games as distinct from ordinary life, operating within what Johan \textcite{huizingaHomoLudensStudy1998} termed the ``magic circle"—a bounded space where different rules and expectations apply. Both Suits and Huizinga highlight the structured nature of play, which is crucial in understanding enjoyment. 

\section{Game Typologies}
Games can be categorised in various ways, reflecting different modes of play and player motivations. The ways to engage with games can range anywhere from competitive, cooperative, solo, casual, exploratory, and role-playing, each incentivising a distinct playstyle.  Caillois and Barash \textcite{cailloisManPlayGames2001} propose four main groups into which different game styles fall: Agon is a group that is constituted of games that allow for competition amongst the players elicit ``Hard Fun" \autocite{lazzaroWhyWePlay2004a} like Chess and such games revolve around mastery and challenge. While, games that utilise luck and uncertainty as one of their core mechanics fall into the category of games called Alea, such games may elicit ``Easy Fun" \autocite{lazzaroWhyWePlay2004a}, for example, Snakes and Ladders. Engaging in role-play or make-believe scenarios, where the player dons a new persona to immerse themselves in fictional universes, such as Dungeons and Dragons come under the category of games called Mimicry. These games offer an ``altered state" \autocite{lazzaroWhyWePlay2004a} of enjoyment proving to be escapist and transformational experiences. Lastly, Ilinx is a category of games that temporarily disrupts the stability of perception and create a sense of panic in an otherwise clear mind, examples of the same include Bungee Jumping.


\subsection{Deterministic vs. Stochastic Games}
Mathematically speaking, games can be broadly classified based on the predictability of their outcomes and the role of chance: deterministic and stochastic games. Deterministic games have entirely predictable outcomes determined by player actions and game rules. In games like Chess, assuming perfect play, the optimal move can be determined from any board configuration, making the game deterministic in its outcome.  Stochastic games, conversely, incorporate elements of uncertainty or randomness, leading to less predictable outcomes. Snakes and Ladders is a prime example of a stochastic game.  Gameplay is infused with randomness through die rolls, making the game's progression probabilistic rather than predetermined.

\section{Moving Beyond Subjectivity}
While the allure of games is universally acknowledged, the nature of enjoyment itself remains inherently subjective \autocite{lazzaroWhyWePlay2004a}. Existing game review systems, as analysed by  \textcite{yangExperientialGoodsNetwork2010a}, often grapple with this subjectivity, relying on consumer feedback inherently limited by subjective opinions and the tendency to focus on ``search attributes"—features readily apparent before playing—rather than ``experiential attributes"—those felt only through gameplay.  

However, this inherent subjectivity does not negate the need for a more systematic and potentially quantifiable approach to understanding game enjoyment. Indeed, to advance game design and analysis, one must strive to bridge the gap between subjective experience and objective analysis. While experiential enjoyment remains inherently variable, enjoyment, rooted in the game design's core mechanics (or the rules), can be approached as a more quantifiable construct. With this theoretical foundation in place, this dissertation focuses on the specific domain of tabletop games, chosen for their tangible nature and the direct player interaction they allow. By focusing on the design elements and rule systems that structure gameplay, this project aims to develop methods for objectively assessing and potentially predicting the level of enjoyment a game's mechanics might elicit, focusing on what can be termed ``mechanical enjoyment"—the enjoyment derived from the inherent design and mechanics of the game system itself rather than reflecting upon a player's subjective immersion and emotional response.  This ``mechanical enjoyment," attributed to the various components of a game's design refers to a quantitative measure that can potentially be assigned to a game to indicate the level of enjoyment or the utility a player derives from the game. 

The academic study of games is a diverse and interdisciplinary field, spanning user experience design, social context analysis, mathematical game theory, and more \autocite{vlachopoulosEffectGamesSimulations2017}. Key perspectives within game studies include narratology, focusing on games as narrative experiences, and ludology, emphasising game rules and structures \autocite{mcmanusNarratologyLudologyCompeting2006a}. While narratology examines the story and narrative elements within games, ludology prioritises the systematic analysis of game mechanics and player interaction with these systems. The framework used in ``Four Keys to More Emotion'', derived from user experience research, further categorises game enjoyment into “Hard Fun”, “Easy Fun”, “Altered State”, and the “People Factor”, highlighting the diverse sources of player engagement and providing a user-centric perspective \autocites{lazzaroWhyWePlay2004a}.

Mathematical analysis provides another crucial lens for examining games. Game theory, a branch of applied mathematics, offers tools to study strategic decision-making in competitive situations \autocite{vonneumannTheoryGamesEconomic1944a}.  Furthermore, methods like combinatorial analysis and, significantly for this research, Markov Chains, have been applied to analyse game mechanics and dynamics.  For instance, \textcite{raposoMathematicalAnalysisRoyal2023a} employed mathematical analysis to investigate the Royal Game of Ur. These diverse analytical approaches, ranging from theoretical frameworks to mathematical modelling, provide a rich toolkit for objectively investigating game mechanics and their impact on player experience. Specifically, within this domain, we will specifically examine the game of Snakes and Ladders.

\section{Snakes and Ladders: A timeless classic}
Snakes and Ladders, far from being a trivial childhood pastime boasts a rich history and a remarkable universality that makes it an ideal case study for understanding fundamental game mechanics and player engagement. Its origins can be traced back to ancient India, where it was known as \textit{Moksha Patam} or \textit{Gyan Chaupar} \autocite{dusautoyWorldEightyGames2024}. The ladders represented virtues like generosity, faith, and humility, while the snakes symbolized vices such as lust, anger, theft, and pride. The ascent and descent on the board mirrored the karmic cycle of life,
illustrating the consequences of good and bad actions in a visually compelling and accessible way. In the earliest board layouts from India, the snakes handedly outnumbered the number of ladders; The nine snakes to four ladders made achieving \textit{moksha} quite hard \autocite{dusautoyWorldEightyGames2024}. These boards usually were arranged in a rectangular fashion, with the most common boards being $8 \times 9$ with 72 tiles.

Over centuries, \textit{Moksha Patam} travelled beyond India, evolving and adapting as it spread across cultures, a journey that resonates with Sautoy’s (2024) broader narrative of how ideas and concepts traverse geographical and cultural boundaries. By the late 19th century, a Westernised version, ``Snakes and Ladders,” emerged in England and quickly gained popularity worldwide. While the moralistic undertones were done away with in its global iteration, the core mechanics of chance, progression, setbacks, and the simple pursuit of a defined goal remained intact.

While often perceived as a simple children's game, Snakes and Ladders has also proven to be a rich subject for academic inquiry, particularly within the field of mathematical game analysis. By as early as 1967,  we saw that the game constitutes an ``interesting example of a Markov chain," amenable to rigorous mathematical modelling \autocite{daykinMarkovChainsSnakes1967a}. Researchers have since employed Markov chains to analyse various aspects of Snakes and Ladders, including the probability distributions of game length \autocite{tunMarkovProcessSnake2021} and the calculation of expected playing time \autocite{althoenHowLongGame1993a, daykinMarkovChainsSnakes1967a}. \textcite{althoenHowLongGame1993a} estimated the average game length to be approximately 39 moves for a $10\times10$ board, through both analytical calculations and computer simulations, providing a benchmark figure for understanding the typical duration of a game of Snakes and Ladders.

Snakes and Ladders, in its simplicity, offers insight into the broader challenges of quantifying game enjoyment. Its mechanics are easily grasped—the roll of a die dictates movement, and predetermined snakes and ladders introduce elements of both fortune and misfortune. However, even within this seemingly straightforward system, players experience various emotions: anticipation with each dice roll, frustration upon encountering a snake, elation when climbing a ladder, and the ultimate satisfaction of reaching the final square. The game's accessibility and widespread familiarity make it an excellent lens to examine how even basic game mechanics, governed by chance and simple rules, can generate engaging and emotionally resonant player experiences. By analysing Snakes and Ladders through the framework of mechanical enjoyment, one can isolate and understand the core design elements that contribute to the enduring appeal of tabletop games and, potentially, games more broadly. Therefore, this dissertation will use Snakes and Ladders as a central case study to explore and quantify mechanical enjoyment.

\section{Dissertation Structure}
Beginning with the establishment of essential frameworks and a review of existing approaches to game analysis and enjoyability in Chapter 1 of which this section is a part, the research moves towards an empirical investigation of game dynamics. In Chapter 2 we employ agent-based simulations to quantify the influence of specific game parameters on how the game duration is affected, thereby altering the experience of playing Snakes and Ladders.  Building upon these empirical findings in Chapter 3, the investigation then advances to explore the impact of another key design element—the size of the game board—through further simulation-based analysis.  In Chapter 4, we develop a Markov model to analytically derive key game metrics and provide a comparative validation against the empirical results from Chapters 2 and 3.  Finally, the dissertation concludes in Chapter 5 by aggregating the insights gained throughout this exploration, discussing their implications for game design and game enjoyment, also suggesting potential avenues for future research and inquiry. 
