\chapter{The Dynamics of Snakes and Ladders}

The objectivity of a set of rules provides a strong foundation to set up the notion of mechanical enjoyability when it comes to various kinds of systems, especially those like table-top games. Within the clearly defined structure of a game's rules, we can begin to analyze and potentially quantify the sources of enjoyment that arise purely from the system's design, independent of individual player preferences or social contexts.  This concept resonates deeply with the notion of the ``Magic Circle'' (Huazinga, 1938), which describes games as existing within a bounded space governed by self-contained rules and conventions. It is within this ``circle'' of rules that mechanical enjoyability takes shape – an inherent quality of the game system itself, derived from its internal logic and the interactions it engenders through its mechanics.

This chapter delves into the mechanics of the classic game \textit{Snakes and Ladders}, aiming to quantify the impact of various game parameters on the overall game dynamics. This is achieved by simulating numerous games while systematically varying parameters. In this chapter, we will investigate two key aspects: firstly, the impact of the \textit{number} of snakes and ladders on the board, and secondly, the effects of varying the \textit{lengths} of these entities. To simplify the analysis and isolate the effects of these parameters, the model reduces the game to its essential elements. This allows us to systematically examine how changes in these parameters affect the distribution of game duration – specifically, the number of moves needed to reach the end state. This distinction in parameter variability helps separate the core game mechanics from the broader gameplay experience.

The average game time is chosen as a primary metric for quantifying these mechanical aspects. Game time, defined as the number of moves required to reach the end state, serves as a readily measurable and intuitively understandable indicator of game dynamics. It directly reflects the efficiency and predictability of the game system in guiding a player towards its objective. For a game like Snakes and Ladders, where the goal is simply to reach the final tile, the number of turns taken to achieve this outcome becomes a crucial measure of the game's mechanical properties. Variations in average game time, as we will explore, can reveal how different configurations of snakes and ladders, governed by the game's rules, alter the overall pace and challenge of the experience.

\section{Setting up the board}

In our pilot study, the game board is modelled as a 10x10 grid, comprising 100 squares, akin to the classic Snakes and Ladders game. The player, represented by an Agent in our model, starts at tile 1, with no requirement to roll a specific number to begin (i.e., no starting condition). The goal state is to reach or exceed tile 100. The Agent's movement is determined by a fair six-sided die roll.

To facilitate a systematic investigation of game dynamics, this research introduces several controllable parameters that define the entities on the board:

\begin{enumerate}
	\item \textbf{Board Size }($BoardSize$): The maximum size of the board in terms of the number of tiles. The board is of the form $m * m$ and there are a total of $m^2$ tiles on the board.
	\item \textbf{Number of Snakes} ($N_{s}$): The total number of snakes on the board.
	\item \textbf{ Number of Ladders } ($N_{l}$): The total number of ladders on the board.
	\item \textbf{Length of Snakes }($L^{i}_{s}$): This parameter determines the length of the $i^{th}$ snake on the board for $i=1,2,... N_{s}$. It dictates the number of tiles the agent is set back when landing on a snake's head.
	\item \textbf{Length of Ladders} ($L^{i}_{l}$): This parameter determines the length of $i^{th}$ ladder on the board for $i=1,2,... N_{l}$. It dictates the number of tiles the agent climbs when encountering a ladder's base.
	\item \textbf{Ladder Position} ($Ladder^{i}_{base/top}$): The position of the $i^{th}$ ladder's terminal ends.
	\item \textbf{Snake Position} ($Snake^{i}_{head/tail}$): The position of the $i^{th}$ snake's terminal ends.
\end{enumerate}

To ensure the board configuration remains valid and avoids conflicts—such as positioning snakes or ladders at invalid tiles where they might extend beyond the board's boundaries—certain constraints are implemented:

\begin{enumerate}
	\item \textbf{Ladder Constraint:} Ladders cannot begin within the $L^{i}_{l}$ tiles of the board to prevent them from extending beyond the game's end. The ladder's starting position therefore becomes:  $$Ladder^{i}_{start} \leq BoardSize - L_{l}$$
	\item \textbf{Snake Constraint:} Snakes cannot begin within the first $L^{i}_{s}$ tiles to avoid their tails going below the starting position. The snake's end therefore becomes: $$Snake^{i}_{start}\geq 1 + L_{s}$$
	\item \textbf{Overlap Constraint:} To maintain game integrity, no terminal ends of a snake or ladder (start or end) can overlap with any part of another snake or ladder. The paths of snakes and ladders can coincide at various points so long as they don't have overlaps at the ends of the entities. If an overlap occurs, the simulation set-up randomly decides whether to remove the overlapping snake or ladder based on a probability of 0.5.
\end{enumerate}

The board generation process involves randomly selecting starting positions for snakes and ladders within these permissible ranges. This is followed by a validation step to resolve any overlaps. This iterative process continues until a valid board configuration is achieved. The number of iterations required to generate a valid board is recorded and can be analysed to understand the complexity of board creation under different parameter settings.

This structured approach to board generation allows us to systematically vary the parameters and study their individual and combined effects on the game dynamics. By analysing the resulting distributions of game durations, this research aims to uncover trends and patterns that reveal the interplay of these factors in shaping the player's experience.

\section{Approaches to Assign Entity Parameters}

To comprehensively explore the dynamics of \textit{Snakes and Ladders}, this research employ different approaches for assigning the key parameters: the number of entities and, in particular, the lengths of snakes and ladders.

\subsection{Varying the Number of Snakes and Ladders}

During the preliminary exploration,  the primary focus is on how the \textit{number} of snakes ($N_S$) and ladders ($N_L$) affects the average game time, while keeping the lengths of these entities consistent across simulations. Using simulated data, the project explores the relationship between different counts of snakes and ladders, maintaining their lengths as variables influenced by board constraints but not systematically varied in this phase. The game is simulated 1000 times for each of the 10 distinct board configurations, varying the number of snakes and ladders independently of each other to observe their isolated and combined impacts.

\subsection{Varying the Lengths of Snakes and Ladders}

To investigate the impact of entity lengths, three distinct approaches for assigning lengths ($L^S_i$ and $L^L_i$) on the game board are deployed. Each approach allows for unique characteristics of the board configuration to facilitate a comparative analysis of game time under varying assumptions:

\begin{enumerate}
	\item \textbf{Fixed Unequal Lengths:} This deterministic approach assigns fixed but unequal lengths to all snakes and ladders. For instance, one might set $L^S_1 = 10$, $L^L_1 = 5$, $L^S_2 = 20$, $L^L_2 = 10$ and so on, ensuring that $L^S_i \neq L^L_i$ for all $i \in [1,N]$, while maintaining consistency in the lengths assigned to entities of the same type across different simulations within this approach. This method provides a baseline for analyzing gameplay outcomes under deterministic length conditions and ensures uniformity across experiments.
	
	\item \textbf{Sampling from Distributions:} To introduce variability in lengths, this approach employs three distinct probability distributions—uniform, normal, and exponential—to sample lengths for each snake and ladder ($L^S_i$ \& $L^L_i$ $\forall i \in [1, N]$).
	
	\begin{enumerate}
		\item \textbf{Uniform distribution: }All valid lengths between 1 and $L_max$ are equally likely, this ensures an unbiased selection across the entire range of lengths, providing a uniform probability for shorter and longer lengths.
		$$P(L=x)=\frac{1}{L_{max}}, \forall x \in \{1,2,\ldots,L_{max}\}$$
		\item \textbf{Normal distribution:} A Normal/Gaussian distribution is characterized by a mean $\mu$ and a standard deviation $\sigma$, for the lengths of snakes and ladders:
		\begin{itemize}
			\item $\mu$ is set to $\frac{L_{max}}{2}$, placing the most likely lengths near the midpoint of the range
			\item $\sigma$ is set to $\frac{L_{max}}{6}$, which suggests that most lengths fall within the range $[\mu-3\sigma, \mu+3\sigma]$
			$$P(L=x) = \frac{1}{\sqrt{2\pi\sigma^2}}{e^{\frac{(x-\mu)^2}{2\sigma^2}}, \forall x\in[1, L_{max}]}$$
		\end{itemize}
		\item \textbf{Exponential distribution: }This emphasizes on the shorter lengths, with the probability of longer lengths decreasing exponentially. The scaling parameter $\lambda$ is set to $\frac{L_{max}}{3}$ ensuring a reasonable spread of values.
		$$P(L=x)={\lambda}e^{{-\lambda}x}, \forall x \in [1, L_{max}]$$
	\end{enumerate}
	
	\item \textbf{Fixed Start and End Points:} This approach diverges from directly controlling the $L_s$ and $L_l$. Instead, it involves assigning randomized $Ladders^i_{base/top}$ and $Snakes^i_{head/tail}$. This approach to the problem introduces another layer of variability by purely focusing on their placement rather than predetermined or sampled lengths.
	For each snake, the $Snake^i_{head}$ is chosen from the range $[2, BoardSize - 1]$ abiding by the snake constraint. While, the $Snake^i_{tail}$ is determined by randomly selecting tile below its starting position, i.e. $$1\leq Snake^i_{tail} < Snake^i_{head}$$ 
	For ladders, the $Ladder^i_{base}$ is chosen randomly from $[2, BoardSize - 1]$ keeping the ladder constraint in check, whilst its $Ladder^i_{top}$ is assigned randomly above its starting position, i.e. $$Ladder^i_{top}>Ladder^i_{base}$$ 
	
	By decoupling length from predetermined distributions, the method accommodates a wider variety of configurations, making it suitable for exploring edge cases in gameplay. The method allows for a high degree of randomness in gameplay and will be used to test the robustness of the study, offering insights into how random placement and implicit lengths impact game duration, difficulty, and variability.
\end{enumerate}


\section{Presenting Findings: Impact of Entity Number and Length on Game Time}

The simulations represent the inherently probabilistic nature of \textit{Snakes and Ladders}, where outcomes are largely determined by the interplay between board design, entity parameters, and the randomness of dice rolls. The analysis of simulation results is presented through distinct visualizations: box plots, trend lines, and heatmaps, offering insights into distribution, trends, and overall patterns, respectively.

\subsection{Distribution of Average Game Times: Varying Number of Entities}

The box plot (Fig. \ref{fig:boxplots}) illustrates the distribution of average game times across various combinations of $N_S$ and $N_L$. Game configurations with extreme differences in the number of snakes and ladders (e.g., $N_S = 10$ and $N_L = 5$) exhibit the highest variability in average game time, observable through wider interquartile ranges and more frequent outliers. A key observation is that configurations with fewer ladders tend to result in longer game times. For instance, configurations with 10 snakes and 5 ladders frequently show average game times above 33 moves, with some extreme outliers nearing 37 moves. Conversely, average game time tends to decrease as the number of ladders increases. This is intuitive, as ladders facilitate faster progress towards the goal state, reducing the number of turns required to complete the game. Lower average game times, especially in configurations with more ladders, often show tighter distributions with fewer outliers, suggesting a more consistent game time. The plot concludes that outliers are more frequent in configurations with more snakes and fewer ladders. These outliers could represent scenarios where luck significantly impacts the game, either by allowing a player to avoid major snakes (resulting in an unusually short game) or by repeatedly encountering snakes (leading to exceptionally long games).

\begin{figure}[th]
	\centering
	\includegraphics[width=0.8\textwidth]{"../Chapter 1/BoxPlots"}
	\caption{\textbf{Distribution of Average Game Times Grouped by $N_S$ and $N_L$:} Box plot shows higher variability in average game time for extreme $N_S$/$N_L$ differences. Fewer ladders correlate with longer game times; outliers indicate luck-dependent game lengths.}
	\label{fig:boxplots}
\end{figure}

%	\subsection{Trend in Average Game Times for Different Configurations}
%	
%	The line graph (Fig. \ref{fig:avg_game_times_trend}) tracks trends in average game times across different board configurations for each combination of snakes and ladders. This visualization compares game performance for each combination as the board layout changes across simulations. A prominent observation is the lack of a consistently dominant configuration. While certain combinations appear to yield consistently higher or lower game times, the lines frequently intersect, indicating that no single combination guarantees the fastest or slowest game across all board layouts. The line chart reveals that configurations with fewer snakes ($N_S = 5, 6$) tend to produce more stable game times, regardless of the number of ladders. These configurations generally exhibit fewer extreme spikes and dips in average game time, indicated by smoother lines on the graph. This suggests that a reduced risk profile in board layout leads to more consistent game durations from run to run.
%	
%	For configurations with higher snake counts ($N_S \geq 9$), noticeable spikes in average game time emerge for certain boards. These spikes suggest that snake placement on specific boards can create significant delays, likely due to frequent encounters with snakes that force players back multiple spaces. In these scenarios, the impact of ladders is diminished, as any advantage gained from a ladder can be negated by a snake encountered later. Configurations with 9 or 10 ladders generally exhibit less fluctuation across boards. While they may not consistently provide the shortest game times, they offer more consistent results by reducing extreme outliers and spikes. This effect also counteracts the impact of high snake counts, generally stabilizing the game mechanics.
%	
%	\begin{figure}[th]
	%		\centering
	%		\includegraphics[width=0.8\textwidth]{"../Chapter 1/Combined_Trends"}
	%		\caption{Trend in Average Game Times for Different Configurations}
	%		\label{fig:avg_game_times_trend}
	%	\end{figure}
%	
\subsection{Interaction Between Number of Snakes and Ladders}

The heatmap (Fig. \ref{fig:heatmap}) provides a visual overview of the interaction between $N_S$ and $N_L$ on average game time. Colors range from dark shades (indicating shorter game times) to bright shades (indicating longer game times). A clear pattern emerges: as the number of ladders increases, the average game time decreases. This trend is most pronounced for higher snake counts ($N_S \geq 9$), where additional ladders significantly reduce game times. The bottom-right corner of the heatmap ($N_S \geq 9, N_L = 10$) shows the shortest game times, reinforcing the idea that ladders effectively mitigate the delays caused by snakes.

The effect of snakes on game time, however, is not strictly linear. For example, increasing the number of snakes from 5 to 6 does not dramatically alter the game time. Yet, when the number of snakes is increased to 9 or 10, game times rise substantially. This suggests a threshold beyond which additional snakes significantly increase the likelihood of players encountering them, thus extending game time considerably. This indicates that while snakes introduce challenges, their negative impact on game duration can be mitigated by providing players with ample ladders to climb back up.

\begin{figure}[th]
	\centering
	\includegraphics[width=0.5\textwidth]{"../Chapter 1/Heatmap"}
	\caption{\textbf{Heatmap of Average Game Times:} Heatmap shows decreasing average game time with increasing $N_L$, especially at higher $N_S$. $N_S$ effect is non-linear; game time significantly increases only beyond a certain $N_S$ threshold.}
	\label{fig:heatmap}
\end{figure}


\subsection{Controlled Approach: Unequal Snake and Ladder Lengths}

This section presents findings from simulations using fixed, unequal lengths for snakes and ladders across 10 different board configurations, with 1000 simulations per configuration. $L^S$ and $L^L$ were  systematically varied in pairs, ensuring $L^S$ and $L^L$ were consistently unequal within each pair type. The bar plot (Fig. \ref{fig:approach1fixedlengthpairsbarplot}) illustrates the average game times for various length pairs. It is evident that average game time generally increases as the difference between snake length and ladder length ($L^S - L^L$) widens. This suggests that when snakes are significantly longer than ladders, players experience more setbacks, contributing to longer average game times. Conversely, no significant impact on game time is observed across pairs where ladder lengths exceed snake lengths ($L^L > L^S$).

\begin{figure}[th]
	\centering
	\includegraphics[width=0.5\textwidth]{"../Chapter 2/withLength/UnequalLengths/approach_1_fixed_length_pairs_barplot"}
	\caption{\textbf{Average Game Times for Fixed Unequal Lengths:}Bar plot shows game time increases with widening snake-ladder length difference ($L_S - L_L$), particularly when snakes are longer. $L_L > L_S$pairs show minimal game time impact.}
	\label{fig:approach1fixedlengthpairsbarplot}
\end{figure}

The frequency distribution plot (Fig. \ref{fig:fixed_lengths_dist} (c)) provides a detailed view of game time distribution for the configuration with the most extreme length disparity ($L^S = 40$ \& $L^L = 20$). The distribution is notably right-skewed, indicating that while most games conclude within a moderate number of moves, occasional runs extend significantly longer. This skew is likely attributable to the inherent randomness of dice rolls and the frequency of agent-snake encounters, even with relatively long ladders present. Comparing configurations with $L^L > L^S$ reveals that those with smaller differences ($L^L - L^S < 20$) exhibit tighter distributions resembling a normal distribution with few outliers. The average game times in these configurations also cluster more closely together, unlike the pronounced spikes observed in configurations with larger length disparities.

\begin{figure}[ht]
	\centering
	\subfloat[]{\includegraphics[width=0.4\textwidth]{"../Chapter 2/withLength/UnequalLengths/game_time_distribution_Snake10_Ladder5"}} 
	\subfloat[]{\includegraphics[width=0.4\textwidth]{"../Chapter 2/withLength/UnequalLengths/game_time_distribution_Snake20_Ladder10"}} 
	\linebreak
	\subfloat[]{\includegraphics[width=0.4\textwidth]{"../Chapter 2/withLength/UnequalLengths/game_time_distribution_Snake40_Ladder20"}} 
	\caption{\textbf{Game Time Distributions for Configurations with $L_S > L_L$:}Histograms of these configurations show right-skewness, indicating occasional longer games. Skew is attributed to dice roll randomness and snake encounters, despite ladders.}
	\label{fig:fixed_lengths_dist}
\end{figure}


\subsection{Using Sampling Distributions for Lengths}

This section investigates the effects of different sampling distributions on snake and ladder lengths ($L^S$ and $L^L$). We explore three statistical distributions—uniform, normal, and exponential—each with $N_S$, $N_L = 10$, from which $L^S$ and $L^L$ are sampled. For each distribution, 1000 games were simulated across 10 different boards. Figure \ref{fig:sampling_dist_avg_times} shows the aggregated average game times across these distributions. The highest average game time results from the exponential sampling method, followed by the normal distribution, with the uniform distribution yielding the lowest average game time. Exponential sampling tends to produce more shorter lengths, with a lower probability of longer entities, suggesting that a higher density of smaller snakes and ladders extends game duration. In contrast, normal distribution, which clusters lengths around a midpoint, results in more stable configurations with lower average game times.

\begin{figure}[ht]
	\centering
	\includegraphics[width=0.5\textwidth]{"/home/ja1peg/Desktop/DissertationDocs/Dissertation/Chapter 2/withLength/FinalSampling/comparative_aggregate_average_game_times.png"} 
	\caption{\textbf{Aggregated Averages of Game Time across the sampling distributions:} Bar plot compares average game times for uniform, normal, and exponential length distributions. Exponential distribution yields highest, normal lowest average game time, suggesting shorter lengths extend game duration.}
	\label{fig:sampling_dist_avg_times}
\end{figure}

Frequency distributions of game times for a fixed board layout under each sampling method (Fig. \ref{fig:sampling_dist_layout_dists}) reveal right-skewness across all three distributions, indicating that while most games finish within a moderate set of moves, outliers leading to longer games are possible. Exponential sampling exhibits the highest variability, possibly due to the prevalence of smaller entities and their random placement. Figure \ref{fig:sampling_dist_board_avg_times} (b) indicates that boards generated using exponential sampling consistently result in higher average game times compared to other methods. Figure \ref{fig:sampling_dist_board_avg_times} (a) shows that uniform distribution results in more consistently lower average game times across different board layouts, albeit with some boards exhibiting higher averages. Normal distribution displays the most variability across the boards played but with consistently low average game times and fewer outliers pushing game times upward.

\begin{figure}[ht]
	\centering
	\subfloat[]{\includegraphics[width=0.3\textwidth]{"/home/ja1peg/Desktop/DissertationDocs/Dissertation/Chapter 2/withLength/FinalSampling/full_distribution_exponential.png"}} 
	\subfloat[]{\includegraphics[width=0.3\textwidth]{"/home/ja1peg/Desktop/DissertationDocs/Dissertation/Chapter 2/withLength/FinalSampling/full_distribution_normal.png"}} 
	\subfloat[]{\includegraphics[width=0.3\textwidth]{/home/ja1peg/Desktop/DissertationDocs/Dissertation/Chapter 2/withLength/FinalSampling/full_distribution_uniform.png}} 
	\caption{\textbf{Game Time Distributions for a Fixed Layout, by Sampling Method:} Histograms for a fixed layout show right-skewness across all sampling methods. Exponential sampling exhibits highest variability, possibly due to smaller entities and placement.}
	\label{fig:sampling_dist_layout_dists}
\end{figure}

\begin{figure}[ht]
	\centering
	\subfloat[]{\includegraphics[width=0.45\textwidth]{"/home/ja1peg/Desktop/DissertationDocs/Dissertation/Chapter 2/withLength/FinalSampling/board_averages_exponential.png"}} 
	\subfloat[]{\includegraphics[width=0.45\textwidth]{"/home/ja1peg/Desktop/DissertationDocs/Dissertation/Chapter 2/withLength/FinalSampling/board_averages_normal.png"}}
	\linebreak
	\subfloat[]{\includegraphics[width=0.45\textwidth]{"/home/ja1peg/Desktop/DissertationDocs/Dissertation/Chapter 2/withLength/FinalSampling/board_averages_uniform.png"}}
	\caption{\textbf{Average Game Time for Each Board, by Sampling Distribution:} The plots show average game time for 10 boards, by sampling distribution. Uniform distribution yields consistently lower averages. Normal distribution shows highest board variability, but low average times. Exponential sampling generally results in higher average times.}
	\label{fig:sampling_dist_board_avg_times}
\end{figure}


\subsection{Randomly Generated Boards: Fixed Start and End Points}

This section examines the effect of randomly generated snakes and ladders based on fixed start and end positions. Positions are generated while adhering to length constraints. This approach introduces greater variability compared to methods where lengths are predetermined. Figure \ref{fig:random_boards_avg_times} (a) illustrates average game times across 10 randomly generated boards. Each bar represents the average game time for a specific board, revealing significant variability in average game times across different board layouts, ranging from approximately 22 to 40 moves. Figure \ref{fig:random_boards_game_dist} (b) displays the frequency distribution of game times across all simulated games, indicating similar variability, as evidenced by random spikes in certain sections and the kernel density line. Certain boards, due to their specific configurations, present varying challenges and opportunities for the agent, leading to a wide range of game durations.

\begin{figure}[ht]
	\centering
	\subfloat[]{\includegraphics[width=0.45\textwidth]{"/home/ja1peg/Desktop/DissertationDocs/Dissertation/Chapter 2/withLength/RandomLength/approach_3_game_time_distribution.png"}}
	\subfloat[]{\includegraphics[width=0.45\textwidth]{"/home/ja1peg/Desktop/DissertationDocs/Dissertation/Chapter 2/withLength/RandomLength/approach\_3\_random\_points.png"}} 
	\caption{\textbf{Analysis of Randomly Generated Boards:} (a) Game time distribution for a random board shows variability within a board. (b) Average game time across 10 random boards varies significantly (22-40 moves), highlighting random placement impact.}
	\label{fig:random_boards_avg_times}
	\label{fig:random_boards_game_dist}
\end{figure}


\section{Analysis}

This chapter explores the impacts of both entity number and length ($L^S$ and $L^L$) on game dynamics in \textit{Snakes and Ladders}. By simulating games under systematically varied parameters across different approaches—static length assignment, statistical length sampling, and randomized start/end points, revealing several key insights.

Firstly, variations in both the \textit{number} and \textit{lengths} of snakes and ladders significantly influence game duration and its variability. Increased numbers of snakes generally prolong games, while more ladders tend to shorten them. However, the relationship is not always linear, with thresholds and interactions between entity types playing a crucial role. For example, while simply increasing snake count does not always linearly increase game time, exceeding a certain density of snakes dramatically extends game duration. Conversely, the positive impact of ladders is more pronounced in mitigating the negative effects of high snake counts.

Secondly, the method of assigning entity lengths introduces another layer of complexity. Deterministic length assignments provide a baseline for understanding game mechanics, while statistical sampling reveals how different length distributions affect game variability and average duration. Exponential distributions, favoring shorter lengths, tend to increase game time and variability, whereas normal distributions offer a more balanced outcome. Uniform distributions, in contrast, result in shorter, more predictable games. Randomly generated boards based on fixed start and end points introduce the highest degree of variability, highlighting the significant impact of entity placement on overall game dynamics.

Thirdly, the presence of outliers and fluctuating trend lines across different board layouts underscores the significant influence of board layout itself. Strategic placement of snakes and ladders can create "traps" or "shortcuts," leading to substantial variations in game duration even within the same parameter configurations.

These findings have valuable implications for game design and balancing. They demonstrate how adjusting the number and lengths of snakes and ladders can fine-tune game difficulty and duration. The choice of length assignment method—deterministic, sampled, or position-based—further allows designers to control the level of variability and unpredictability in gameplay. For games with mechanics similar to \textit{Snakes and Ladders}, these insights can inform the creation of engaging experiences with carefully calibrated challenge and playtime.

Additionally, this chapter's exploration of entity number and length, while illuminating, represents just one facet of game design parameterisation. The observed sensitivity of game dynamics to these entity-level adjustments naturally prompts further investigation into other fundamental game design elements. In the subsequent chapter, we pivot our focus to another core parameter: the \textit{scale} of the game board itself. By systematically varying board dimensions, while holding entity characteristics constant, we aim to unravel how board size, as a determinant of game space and traversal distance, independently shapes game hardness, duration, and overall player experience. This shift in focus, from entity-level parameters to board-level dimensions, represents the next logical step in our systematic parameterisation of Snakes and Ladders gameplay, allowing for a more holistic understanding of the game's design space.
