\documentclass[12pt]{report}
\usepackage[utf8]{inputenc}
\usepackage{xcolor}
\usepackage[margin=1in]{geometry}
\usepackage{soul} 
\usepackage{apacite}
\usepackage{graphicx}
\usepackage{xurl}
\usepackage{subcaption}
\captionsetup{compatibility=false}

\usepackage{fancyhdr}
\pagestyle{fancy}

\lhead{Research Project}
\rhead{Draft Chapter 2}
\cfoot{\thepage}
\renewcommand{\headrulewidth}{0.4pt}
\renewcommand{\footrulewidth}{0.4pt}

\title{Alternate Review Systems:\\ Quantifying Enjoyability in Table-top Games}
\author{Jai Bakshi \\ 21060322069}
\date{November 2024}

\begin{document}
	\maketitle
	\setcounter{chapter}{-1}
	\chapter{Introduction}
	From the simple delight of childhood board games to the intricate strategies of modern tabletop experiences, games hold a fundamental appeal for humanity. As Peter Gray (2017)  argues, play is not merely frivolous pastime but a powerful vehicle for learning and development, deeply ingrained in our nature. Games, in their essence, are structured systems that invite players to engage in artificial conflicts governed by predefined rules, ultimately leading to quantifiable outcomes (Puentedura, n.d.). This act of play, this engagement within a rule-bound system, is where the potential for enjoyment resides. Understanding and quantifying this ‘enjoyment’ becomes a complex undertaking. It is influenced by a myriad of factors, ranging from individual preferences and social dynamics to the inherent design and mechanics of the game itself. In the context of game studies and design, it becomes crucial to move beyond subjective impressions and explore methods to systematically analyse and potentially quantify the elements that contribute to a game's enjoyability.
	
	This dissertation focuses on the specific domain of table-top games – a rich and tangible space for exploring game mechanics and player interaction. Tabletop games, encompassing board games, card games, and dice games, offer a unique lens through which to examine the relationship between game design and player experience. Their tangible nature, the direct manipulation of components, and the face-to-face social interaction create a particularly fertile ground for investigating the sources of game enjoyment. Within this exploration, we must consider the philosophical underpinnings of what constitutes a game. Bernard Suits, in his seminal work The Grasshopper: Games, Life and Utopia (1978), provides a valuable framework. Suits introduces the concept of the ``lusory attitude,'' the willing acceptance of constitutive rules to engage in activity aimed at achieving a specific state of affairs (the lusory goal), where such rules prohibit the most efficient means of achieving that state. This ``lusory attitude'' is central to understanding games as distinct from ordinary life, operating within what Johan Huizinga termed the ``magic circle''—a bounded space where different rules and expectations apply. By embracing this perspective, we can begin to dissect the intricate relationship between game mechanics, player engagement, and the elusive quality of enjoyability. For the purposes of this research, we will distinguish between two key aspects of game enjoyment: mechanical enjoyability – the enjoyment derived from the inherent design and mechanics of the game system itself, and experiential enjoyability – the fluctuating enjoyment stemming from player decisions, social interactions, and the unfolding narrative of a particular game session.
	
	\section{Moving Beyond Subjectivity}
	While the allure of games is universally acknowledged, the nature of enjoyment itself remains inherently subjective. As Nicole Lazzaro (2004) emphasizes in her examination of “Four Keys to More Emotion Without Story,” factors like social interaction (``People Factor'') and individual player preferences significantly shape the overall gaming experience. Consequently, assessing game enjoyability is not as straightforward as evaluating objective features. Existing game review systems, as analysed by Yang and Mei (2010), often grapple with this subjectivity. These systems, while providing valuable consumer feedback, are inherently limited by their reliance on subjective opinions and the tendency to focus on “search attributes” – features readily apparent before playing – rather than “experiential attributes” – those felt only through gameplay. Furthermore, Yang and Mei’s research reveals that negative reviews can disproportionately influence perceptions, and the network effect, where shared experiences amplify enjoyment (or dissatisfaction), further complicates objective assessment.
	
	This inherent subjectivity does not negate the need for a more systematic and potentially quantifiable approach to understanding game enjoyment. Indeed, to advance game design and analysis, we must strive to bridge the gap between subjective experience and objective analysis. This dissertation posits that while experiential enjoyability remains inherently variable, mechanical enjoyability, rooted in the game's core mechanics, can be approached as a more quantifiable construct. By focusing on the design elements and rule systems that structure gameplay, we aim to develop methods for objectively assessing and potentially predicting the level of enjoyment a game's mechanics might elicit.
	
	\section{Snakes and Ladders: A timeless classic}
	To ground our exploration of game enjoyability in a tangible example, this dissertation will reference the game of Snakes and Ladders. As Marcus du Sautoy (2023) might illuminate in his works exploring the mathematical underpinnings of games, Snakes and Ladders is far from being a trivial childhood pastime; it boasts a rich history and a remarkable universality that makes it an ideal case study for understanding fundamental game mechanics and player engagement.  Its origins can be traced back to ancient India, where it was known as \textit{Moksha Patam} or \textit{Gyan Chaupar}.  Reflecting insights often found in du Sautoy's discussions on the history of games and the mathematics surrounding them, this game, believed to have emerged as early as the 2nd century BC, was not merely entertainment but served as a moral and didactic tool. The ladders represented virtues like generosity, faith, and humility, while the snakes symbolized vices such as lust, anger, theft, and pride.  The ascent and descent on the board mirrored the karmic cycle of life, illustrating the consequences of good and bad actions in a visually compelling and accessible way.
	
	Over centuries, \textit{Moksha Patam} traveled beyond India, evolving and adapting as it spread across cultures, a journey that resonates with du Sautoy's broader narratives of how ideas and concepts traverse geographical and cultural boundaries. By the late 19th century, a Westernized version, "Snakes and Ladders," emerged in England and quickly gained popularity worldwide. While the overt moralistic undertones diminished in its global iteration, the core mechanics of chance, progression, setbacks, and the simple pursuit of a defined goal remained intact.  This enduring appeal across diverse cultures and time periods, echoing themes of universality often explored by du Sautoy in mathematical and historical contexts, underscores the game's ability to tap into fundamental aspects of human engagement and enjoyment.
	
	Snakes and Ladders, in its simplicity, offers a microcosm of the broader challenges in quantifying game enjoyability.  Its mechanics are easily grasped – the roll of a die dictates movement, and predetermined snakes and ladders introduce elements of both fortune and misfortune.  Yet, even within this seemingly straightforward system, players experience a range of emotions: anticipation with each dice roll, frustration upon encountering a snake, elation when climbing a ladder, and the ultimate satisfaction of reaching the final square.  The game’s accessibility and widespread familiarity make it an excellent lens through which to examine how even basic game mechanics, governed by chance and simple rules, can generate engaging and emotionally resonant player experiences. By analyzing Snakes and Ladders through the framework of mechanical and experiential enjoyability, we can begin to isolate and understand the core design elements that contribute to the enduring appeal of tabletop games, and potentially, games more broadly.
	
	
\end{document}
